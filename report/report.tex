%%%%%%%%%%%%%%%%%%%%%%%%%%%%%%%%%%%%%%%%%%%%%%%
%%% DISCLAIMER: The original template for this
%%% file can be found at:
%%%
%%% https://www.overleaf.com/latex/templates/report-template-stima-laborations-overleaf-v1-dot-0/jtctxkqjnjdz
%%%
%%% Template for lab reports for CS341 @ IITB
%%%%%%%%%%%%%%%%%%%%%%%%%%%%%%%%%%%%%%%%%%%%%%%

%%%%%%%%%%%%%%%%%%%%%%%%%%%%%% Sets the document class for the document
% Openany is added to remove the book style of starting every new chapter on an odd page (not needed for reports)
\documentclass[11pt, swedish, openany]{book}

%%%%%%%%%%%%%%%%%%%%%%%%%%%%%% Loading packages that alter the style
\usepackage[]{graphicx}
\usepackage[]{color}
\usepackage{alltt}
\usepackage[T1]{fontenc}
\usepackage[utf8]{inputenc}
\usepackage{float}
\usepackage{multirow}
\usepackage{tablefootnote}
\usepackage{wrapfig}
\usepackage{amsmath}
\usepackage{placeins}

\setcounter{secnumdepth}{3}
\setcounter{tocdepth}{3}
\setlength{\parskip}{\smallskipamount}
\setlength{\parindent}{0pt}

% Set page margins
\usepackage[top=100pt,bottom=100pt,left=68pt,right=66pt]{geometry}

% Package used for placeholder text
\usepackage{lipsum}

% Prevents LaTeX from filling out a page to the bottom
\raggedbottom

% Adding both languages, Swedish and English, so they can be used intermittently in for example abstracts.
\usepackage[swedish, english]{babel}

% All page numbers positioned at the bottom of the page
\usepackage{fancyhdr}
\fancyhf{} % clear all header and footers
\fancyfoot[C]{\thepage}
\renewcommand{\headrulewidth}{0pt} % remove the header rule
\pagestyle{fancy}

% Changes the style of chapter headings
\usepackage{titlesec}
\titleformat{\chapter}{\normalfont\LARGE\bfseries}{Part \thechapter:}{0.5em}{}
% Change distance between chapter header and text
\titlespacing{\chapter}{0pt}{50pt}{2\baselineskip}

% Adds table captions above the table per default
\usepackage{float}
\floatstyle{plaintop}
\restylefloat{table}

% Adds space between caption and table
\usepackage[tableposition=top]{caption}

% Adds hyperlinks to references and ToC
\usepackage{hyperref}
\hypersetup{hidelinks, linkcolor = blue} % Changes the link color to black and hides the hideous red border that usually is created

% If multiple images are to be added, a folder (path) with all the images can be added here
\graphicspath{{images/}}

% Separates the first part of the report/thesis in Roman numerals
\frontmatter


%%%%%%%%%%%%%%%%%%%%%%%%%%%%%% Starts the document
\begin{document}

%%% Selects the language to be used for the first couple of pages
\selectlanguage{english}

%%%%% Adds the title page
\begin{titlepage}
    \clearpage\thispagestyle{empty}
    \centering
    \vspace{2cm}

    % Titles
    {\large CS341: Computer Architecture Lab \par}
    \vspace{4cm}
    {\Huge \textbf{Lab Assignment 4} \par}
    \vspace{0.2cm}
    {\huge \textbf{Report} \par}
    \vspace{1.2cm}
    {\normalsize Devansh Jain \texttt{(190100044)} \par}
    \vfill
    {\includegraphics[scale=0.30]{iitb_logo/iitb_logo.eps} \par}
    \vspace{0.5cm}
    {\normalsize
        Department of Computer Science and Engineering \\
        Indian Institute of Technology Bombay \par}
    {\normalsize 2021-2022 \par}

\end{titlepage}

% Adds a table of contents
\tableofcontents{}

\clearpage

% Uncomment the following three rows for a table of figures and/or tables as they are not needed for lab reports
% \listoffigures
% \clearpage
% \listoftables

\mainmatter
\setcounter{chapter}{-1}

%%%%%%%%%%%%%%%%%%%%%%%%%%%%%%%%%%%%%%%%%%%%%%%%
%% Abstract
%%%%%%%%%%%%%%%%%%%%%%%%%%%%%%%%%%%%%%%%%%%%%%%%
\chapter*{\centerline{Abstract}}
Summarize the objective of the lab, what experiments you have conducted, what were the results that you have obtained in a clear and concise manner. Numbers matter, not just words only, for ex. \emph{very high}, \emph{slow} etc.

%%%%%%%%%%%%%%%%%%%%%%%%%%%%%%%%%%%%%%%%%%%%%%%%
%% Part 0: Getting Things Ready
%%%%%%%%%%%%%%%%%%%%%%%%%%%%%%%%%%%%%%%%%%%%%%%%
\chapter{Getting Things Ready}


%%%%%%%%%%%%%%%%%%%%%%%%%%%%%%%%%%%%%%%%%%%%%%%%
%% Part 1: Profiling with VTune
%%%%%%%%%%%%%%%%%%%%%%%%%%%%%%%%%%%%%%%%%%%%%%%%
\chapter{Profiling with VTune}


%%%%%%%%%%%%%%%%%%%%%%%%%%%%%%%%%%%%%%%%%%%%%%%%
%% Part 2: Simulating with ChampSim
%%%%%%%%%%%%%%%%%%%%%%%%%%%%%%%%%%%%%%%%%%%%%%%%
\chapter{Simulating with ChampSim}


\end{document}
